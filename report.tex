\documentclass[11pt]{article}
\usepackage{amsmath}
\usepackage{amsfonts}
\usepackage{amsthm}
\usepackage[utf8]{inputenc}
\usepackage[margin=0.75in]{geometry}

\title{CSC111 Winter 2026 Project 1}
\author{DAEUN LEE and ATTICUS LI}
\date{\today}

\begin{document}
\maketitle

\section*{Running the game}
Run adventure.py

\section*{Game Map}
Example game map below (edit it to show your actual game map):

\begin{verbatim}
 -1  2  3
 -1  1 -1
  6  5  4
\end{verbatim}

Starting location is: 1

\section*{Game solution}
List of commands:

\section*{Lose condition(s)}
Description of how to lose the game: If the player takes too long (as in, types in CERTAINOSAEIDHGUOHGUOHG amount of 
commands)

List of commands:

Which parts of your code are involved in this functionality:

% Copy-paste the above if you have multiple lose conditions and describe each possible way to lose the game

\section*{Inventory}

\begin{enumerate}
\item All location IDs that involve items in the game: 1, 2, 3, 4, 5, 6

\item Item data:
\begin{enumerate}
    \item For Item 1: 
    \begin{itemize}
    \item Item name: Phone
    \item Item start location ID: 1
    \end{itemize}
        \item For Item 2:
    \begin{itemize}
    \item Item name: Laptop Charger
    \item Item start location ID: 2
    \end{itemize}
        \item For Item 3:
    \begin{itemize}
    \item Item name: Lucky Mug
    \item Item start location ID: 3
    \end{itemize}
        \item For Item 4:
    \begin{itemize}
    \item Item name: USB Drive
    \item Item start location ID: 4
    \end{itemize}
        \item For Item 5:
    \begin{itemize}
    \item Item name: Gloves
    \item Item start location ID: 5
    \end{itemize}
        \item For Item 6:
    \begin{itemize}
    \item Item name: Powerbank
    \item Item start location ID: 6
    \end{itemize}
    % Copy-paste the above if you have more items, to list ALL items
\end{enumerate}

    \item Exact command(s) that should be used to pick up an item (choose any one or more items for this example), and the command(s) used to use/drop the item (can copy the list you assigned to \texttt{inventory\_demo} in the \texttt{simulation.py} file)
    \item Which parts of your code (file, class, function/method) are involved in handling the \texttt{inventory} command:
    
    In adventure.py, the function display\_items() is the helper function in the main function that searches through the current inventory list (an instance attribute in the AdventureGame class) and prints each object Item (from game\_entities.py) which includes the item name and how many points can be obtained by taking this item.
\end{enumerate}

\section*{Score}
\begin{enumerate}

    \item Briefly describe the way players can earn score in your game. Include the first location in which they can increase their score, and the exact list of command(s) leading up to the score increase:
    
    Players can earn scores by collecting items around the map in different locations. The first location they can find an item in is the starting location 2, Robarts 2F - Dining Area, or location 5, Oak House, depending on where the player decides to go first. When the player runs the game, typing "go north" will direct the player to location 2, where they can pick up the Lucky Mug, which will add 100 points to their total score. If the player types "go south", it will direct the player to location 5, where they can pick up gloves, which will add 50 points to their total score.  
    
    \item Copy the list you assigned to \texttt{scores\_demo} in the \texttt{simulation.py} file into this section of the report:


    \item Which parts of your code (file, class, function/method) are involved in handling the \texttt{score} functionality:
    
    In adventure.py, there is a function take\_item() in the main function. When a player enters a location with an item for the first time, this function will not only add an item to the inventory list, it will also add points to the player's total score (an instance attribute in the AdventureGame class) depending on which item the player finds.
    
    
\end{enumerate}

\section*{Enhancements}
\begin{enumerate}
    \item Describe your enhancement \#1 here
    \begin{itemize}
        \item Brief description of what the enhancement is (if it's a puzzle, also describe what steps the player must take to solve it):
        \item Complexity level (choose from low/medium/high):
        \item Reasons you believe this is the complexity level (e.g., mention implementation details, how much code did you have to add/change from the baseline, what challenges did you face, etc.)
        \item Name the parts of the code which are involved in this enhancement
        \item Copy the list you assigned to \texttt{enhancements\_demo} in the \texttt{simulation.py} file into this section of the report:
    \end{itemize}

    % Uncomment below section if you have more enhancements; copy-paste as needed
    %\item Describe your enhancement here
    %\begin{itemize}
    %    \item Basic description of what the enhancement is:
    %    \item Complexity level (low/medium/high):
    %    \item Reasons you believe this is the complexity level (e.g., mention implementation details)
    %    \item Name the parts of the code which are involved in this enhancement
    %    \item Copy the list you assigned to \texttt{enhancements\_demo} in the \texttt{simulation.py} file into this section of the report:
    %\end{itemize}
\end{enumerate}


\end{document}
